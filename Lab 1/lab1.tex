\documentclass[11pt]{article}
\usepackage{amsmath,amssymb,amsthm}
\usepackage{algorithm}
\usepackage[noend]{algpseudocode} 

%---enable russian----

\usepackage[utf8]{inputenc}
\usepackage[russian]{babel}

% PROBABILITY SYMBOLS
\newcommand*\PROB\Pr 
\DeclareMathOperator*{\EXPECT}{\mathbb{E}}


% Sets, Rngs, ets 
\newcommand{\N}{{{\mathbb N}}}
\newcommand{\Z}{{{\mathbb Z}}}
\newcommand{\R}{{{\mathbb R}}}
\newcommand{\Zp}{\ints_p} % Integers modulo p
\newcommand{\Zq}{\ints_q} % Integers modulo q
\newcommand{\Zn}{\ints_N} % Integers modulo N

% Landau 
\newcommand{\bigO}{\mathcal{O}}
\newcommand*{\OLandau}{\bigO}
\newcommand*{\WLandau}{\Omega}
\newcommand*{\xOLandau}{\widetilde{\OLandau}}
\newcommand*{\xWLandau}{\widetilde{\WLandau}}
\newcommand*{\TLandau}{\Theta}
\newcommand*{\xTLandau}{\widetilde{\TLandau}}
\newcommand{\smallo}{o} %technically, an omicron
\newcommand{\softO}{\widetilde{\bigO}}
\newcommand{\wLandau}{\omega}
\newcommand{\negl}{\mathrm{negl}} 

% Misc
\newcommand{\eps}{\varepsilon}
\newcommand{\inprod}[1]{\left\langle #1 \right\rangle}


\newcommand{\handout}[5]{
	\noindent
	\begin{center}
		\framebox{
			\vbox{
				\hbox to 5.78in { {\bf Научно-исследовательская практика} \hfill #2 }
				\vspace{4mm}
				\hbox to 5.78in { {\Large \hfill #5  \hfill} }
				\vspace{2mm}
				\hbox to 5.78in { {\em #3 \hfill #4} }
			}
		}
	\end{center}
	\vspace*{4mm}
}

\newcommand{\lecture}[4]{\handout{#1}{#2}{#3}{Scribe: #4}{Линейные сравнения #1}}

\newtheorem{theorem}{Теорема}
\newtheorem{lemma}{Лемма}
\newtheorem{definition}{Определение}
\newtheorem{corollary}{Следствие}
\newtheorem{fact}{Факт}

% 1-inch margins
\topmargin 0pt
\advance \topmargin by -\headheight
\advance \topmargin by -\headsep
\textheight 8.9in
\oddsidemargin 0pt
\evensidemargin \oddsidemargin
\marginparwidth 0.5in
\textwidth 6.5in

\parindent 0in
\parskip 1.5ex

\begin{document}
	
	\lecture{}{Лето 2020}{}{Меркулова Ольга}
	

	начнем с выражения 3 = НОД(9, 30) как линейной комбинации 9 и 30. Находится сразу или алгоритмом Евклида, т.к. $3 = 9(-3) + 30 \cdot 1$, то $$ 21 = 7 \cdot 3 = 9(-21) - 30(-7)$$
	
	Таким образом, $x = -21$, $y = -7$ удовлетворяют уравнению Диофанта и, следовательно, все решения рассматриваемого сравнения должны быть найдены по формуле $$x = -21 + \frac{30}{3}t = -21 + 10t$$
	
	Целые числа $x = -21 + 10t$, где $t = 0, 1, 2$ являются несравнимыми  по модулю 30 (но все они сравнимы по модулю 10); таким образом, мы получаем несогласованные решения 
	
	\begin{center}
		$x \equiv -21$(mod 30), $x \equiv -11$(mod 30), $x \equiv -1$(mod 30)
	\end{center}
	
	или, если вы предпочитаете положительные числа, $x \equiv 9, 19, 29$(mod 30).\\
	
	Рассмотрев одно линейное сравнение, можно обратиться к задаче решения системы	
	\begin{center}
		$a_{1}x \equiv b_{1}$(mod $m_{1}$), $a_{2}x \equiv b_{2}$(mod $m_{2}$), ... ,$a_{r}x \equiv b_{r}$(mod $m_{r}$)
	\end{center}
	одновременных линейных сравнений. Будем считать, что модули $m_{k}$, попарно простые. Очевидно, что система не имеет никакого решения, если каждое индивидуальное сравнение не разрешимо; т.е, если $d_{k} \mid b_{k}$ для каждого $k$, где $d_{k} =$ НОД$(a_{k}, m_{k})$. Когда эти условия выполнены коэффициент $d_{k}$ может быть исключен из $k$-ого сравнения для получения новой системы (имеющей тот же набор решений, что и исходная),
	
	\begin{center}
		$a'_{1}x \equiv b'_{1}$()mod $n_{1}$), $a'_{2}x \equiv b'_{2}$(mod $n_{2}$), ... , $a'_{r}x \equiv b'_{r}$(mod $n_{r}$)  
	\end{center}
	
	где $n_{k} = m_{k}\mid d_{k}$ и $НОД(n_{i},n_{j}) = 1$ для $i \ne j$; также, НОД$(a'_{i}, n_{i}) = 1$.\\
	Решения отдельных сравнений принимают форму
	
	\begin{center}
		$x \equiv c_{1}$(mod $n_{1}$), $x \equiv c_{2}$(mod $n_{2}$), ... , $x \equiv c_{r}$(mod $n_{r}$)
	\end{center}
	
	Таким образом, задача сводится к нахождению совместного решения системы сравнений более простого типа.\\
	Вид задачи, которая может быть решена путем одновременных сравнений, имеет долгую историю, она появилась в китайской литературе еще в первом веке н. э. Сунь-Цзы задал: найдите число, которое оставляет остатки 2, 3, 2 при делении на 3, 5, 7 соответственно.(Такие математические головоломки никоим образом не ограничиваются одной культурной сферой; в самом деле, та же самая проблема встречается в Introductio Arithmeticae греческого математика Никомаха, около 100 г. н.э.) В честь их раннего вклада правило получения решения обычно называется Китайской теоремой об остатках.\\
	
	\begin{theorem}
		(Китайская теорема об остатках). \textit{Пусть $n_{1}$, $n_{2}$, ... , $n_{r}$ $-$ положительные целые числа, такие что $НОД(n_{i}, n_{j}) = 1$ для $i /ne j$. Тогда система линейных сравнений}
			\begin{center}
				$x \equiv a_{1}$(mod $n_{1}$),\\
				$x \equiv a_{2}$(mod $n_{2}$),\\
				...\\
				$x \equiv a_{r}$(mod $n_{r}$),\\
			\end{center}
			
		\textit{имеет совместное решение, которое является уникальным по модулю $n_{1}$, $n_{2}$, ... , $n_{r}$. } 
	\end{theorem}
	\begin{proof}
		 начнем с формирования произведения $n = n_{1}n_{2}...n_{r}$. Для каждого $k = 1,2,...,r$, пусть $$N_{k} = n\mid n_{k} = n_{1} ... n_{k-1}n_{k+1} ... n_{r};$$ другими словами, N - это произведение всех целых чисел $n_{i}$ причем
		 коэффициент $n_{k}$ опущен. По гипотезе, $n_{i}$ попарно просты, так что  НОД$(N_{k}, n_{k}) = 1$. 
		 Согласно теории единого линейного сравнения, возможно решить сравнение $N_{k}x \equiv 1$(mod $n_{k}$); назовем единственное решение $x_{k}$. Наша цель состоит в том, чтобы доказать, что целое число $$\overline{x} = a_{1}N_{1}x_{1} + a_{2}N_{2}x_{2} + ... + a_{r}N_{r}x_{r} $$ является совместным решением данной системы.\\ Во-первых, следует заметить, что $N_{i} \equiv 0$(mod $n_{k}$) для $i \ne k$, т.к. $n_{k} \mid N_{i}$ в данном случае. В результате получается, что
		 \begin{center} 
		 $\overline{x} = a_{1}N_{1}x_{1} + ... + a_{r}N_{r}x_{r} \equiv a_{k}N_{k}x_{k}$(mod $n_{k}$).
		 \end{center}
		 Но целое число x, было выбрано для удовлетворения сравнения $N_{k}x \equiv 1$(mod $n_{k}$), из чего следует
		 \begin{center} 
		 $\overline{x} \equiv a_{k}\cdot 1 \equiv a_{k}$(mod $n_{k}$).
		 \end{center}
		 Это показывает, что решение данной системы сравнений
		 существует.\\
		 Что касается утверждения об уникальности, предположим, что $x'$ - это любое другое
		 целое число, удовлетворяющее этим сравнениям. Тогда
		 \begin{center} 
		 $\overline{x}\equiv a_{k} \equiv x'$(mod $n_{k}$), $k = 1,2,...,r$
		 \end{center}
		  и так $n_{k} \mid x''' - x'$ для каждого значения $k$. Поскольку НОД$(n_{i},n_{j}) = 1$ Следствие 2 из Теоремы 2-5 дает нам ключевой момент, что $n_{1}, n_{2},...,n_{r} \mid \overline{x} - x'$; следовательно, $\overline{x} \equiv x' $(mod $n$). Китайская теорема об остатках доказана.  	
	\end{proof}
	\textbf{Пример 4-8}\\
		Задача, поставленная Сунь-Цзы, соответствует системе трех сравнений
			\begin{center}
			$x \equiv 2$(mod 3),\\
			$x \equiv 3$(mod 5),\\
			$x \equiv 2$(mod 7).	
			\end{center}
		В нотации теоремы 4-8 мы имеем $n = 3 \cdot 5 \cdot 7 = 105$ и $$N_{1} = n/3 = 35, N_{2} = n/5 = 21, N_{3} = n/7 = 15.$$ Теперь линейные сравнения 
			\begin{center}	
			$35x \equiv 1$(mod 3), $21x \equiv 1$(mod 5), $15x \equiv 1$(mod 7)
			\end{center}
		удовлетворяются при $x_{1} = 2$, $x_{2} = 1$, $x_{3} = 1$ соответственно. Таким образом, решение системы задается следующим образом: $$\overline{x} = 2 \cdot 35 \cdot 2 + 3 \cdot 21 \cdot 1 + 2 \cdot 15 \cdot 1 = 233.$$ По модулю 105 получаем единственное решение $\overline{x} = 233 = 23$(mod 105).\\
	\textbf{Пример 4-9}\\
		Для второго примера давайте решим линейное сравнение
		\begin{center} 
		$17x \equiv 9$(mod 276).
		\end{center}
		Ввиду $276 = 3 \cdot 4 \cdot 23$ это эквивалентно нахождению решения системы сравнений\\
		\begin{center}
			\begin{minipage}{0.2\textwidth}
				\begin{flushleft}
					$17x \equiv 9$(mod 3)\\
					$17x \equiv 9$(mod 4)\\
					$17x \equiv 9$(mod 23)\\ 
				\end{flushleft}
			\end{minipage}
		или
			\begin{minipage}{0.2\textwidth}
				\begin{flushright}
					$x \equiv 0$(mod 3)\\
					$x \equiv 1$(mod 4)\\
					$17x \equiv 9$(mod 4)
				\end{flushright}
			\end{minipage}	
		\end{center}
		Заметим, что если $x \equiv 0$(mod 3), то $x = 3k$ для любого целого числа $k$. Подставим во второе сравнение системы и получим
		\begin{center}	
		$3k \equiv 1$(mod 4).
		\end{center} 
		Умножение обеих сторон этой конгруэнтности на 3 дает нам
		\begin{center} 
		$k \equiv 9k \equiv 3$(mod4),
		\end{center} 
		то есть $k = 3 + 4j$, где j - целое число, тогда $$x = 3(3 + 4j) = 9 + 12j.$$ Для $x$, чтобы удовлетворять последнему сравнению, мы должны иметь
		\begin{center} 
		$17(9 + 12j) \equiv 9$(mod 23)
		\end{center} 
		или $204j \equiv -144$(mod 23), что сводится к $3j \equiv 6$(mod 23); то есть, $j\equiv 2$(mod 23). Это дает $j = 2 + 23t$, $t$ - целое число, откуда $$x = 9 + 12(2 + 23t) = 33 + 276t.$$ В целом, $x \equiv 33$(mod 276) приводит к решению системы сравнений и, в свою очередь, решением задачи является $17х \equiv 9$(mod 276).
	\begin{center}
		\textbf{ЗАДАЧИ 4.4}
	\end{center}
	\begin{enumerate}
	\item Решите следующие линейные сравнения:
		\begin{enumerate}
			\item $25x \equiv 15$(mod 29).
			\item $5x \equiv 2$(mod 26).
			\item $6x \equiv 15$(mod 21).
			\item $36x \equiv 8$(mod 102).
			\item $34x \equiv 60$(mod 98).
			\item $140x \equiv 133$(mod 98). [\textit{Подсказка:} НОД$(140,301) = 7$.]
		\end{enumerate}
	\item Используя сравнения,решите приведенный ниже диофантовые уравнения:
		\begin{enumerate}
			\item $4x + 51y = 9$. [\textit{Подсказка:} $4x \equiv 9$(mod 51) даёт $x = 15 + 51t$, пока $51y \equiv 9$(mod 4) даёт $y = 3 + 4s$. Найдите связь между $s$ и $t$.]
			\item $12x + 25y = 331$.
			\item $5x - 53y = 17$.	
		\end{enumerate}
	\item Найдите все решения линейного сравнения $3x - 7y \equiv 11$(mod 13).
	\item Решите каждый из следующих наборов одновременных сравнений:
		\begin{enumerate}	
		\item $x \equiv 1$(mod 3), $x \equiv 2$(mod 5), $x \equiv 3$(mod 7)
		\item $x \equiv 5$(mod 11), $x \equiv 14$(mod 29), $x \equiv 15$(mod 31)
		\item $x \equiv 5$(mod 6), $x \equiv 4$(mod 11), $x \equiv 3$(mod 17)
		\item $2x \equiv 1$(mod 5), $3x \equiv 9$(mod 6), $5x \equiv 9$(mod 11)
		\end{enumerate}
	\item Решите линейное сравнение $17x \equiv 3$(mod $2 \cdot 3 \cdot 5 \cdot 7$), используя систему
	\begin{center}
		$17x \equiv 3$(mod 2), $17x \equiv 3$(mod 3), $17x \equiv 3$(mod 5), $17x \equiv 3$(mod 7). 
	\end{center}
	\item Найдите такое наименьшее целое число $a > 2$, что
	\begin{center}
		$2\mid a$, $3\mid a + 1$, $4\mid a +2$, $5\mid a +3$, $6\mid a + 4$.
	\end{center}
	\item
		\begin{enumerate} 
			\item Получите три последовательных целых числа, каждое из которых имеет квадратный коэффициент. [\textit{Подсказка:} Найдите такое целое число $a$, что $2^{2}\mid a$, $3^{2}\mid a + 1$, $5^{2}\mid a + 2$.]
			\item Получите три последовательных числа, первое из которых делится на квадрат, второе - на куб, третье - на чевертую степень.
		\end{enumerate}
	\item (Брахмагупта, 7 век н. э.). Когда из корзины берут 2, 3, 4, 5, 6 яиц за один раз, то в ней остаётся 1, 2, 3, 4, 5 яйца соответственно. Когда их вынимают по 7 штук за раз, ничего не остается. Найдите наименьшее количество яиц, которое могло бы содержаться в корзине.
	\item Проблема корзины яиц часто формулируется в следующей форме: одно яйцо остается, когда из корзины берут 2, 3, 4, 5 или 6 яиц за один раз; но никаких яиц не останется, если их возьмут 7 за один раз. Найдите наименьшее количество яиц, которое могло быть в корзине.
	\item (Древняя Китайская Задача). Банда из 17 пиратов украла мешок с золотыми монетами. Когда они попытались разделить состояние на части, осталось 3 монеты. В завязавшейся драке из-за того, кому достанутся лишние монеты, один пират был убит. Богатство было перераспределено, но на этот раз при равном дележе осталось 10 монет. Снова возник спор, в ходе которого был убит еще один пират. Но теперь все состояние было равномерно распределено между выжившими. Какое наименьшее количество монет могло быть украдено?
	\item Докажите, что сравнения
	\begin{center}
		\begin{minipage}{0.2\textwidth}
			\begin{flushleft}
				$x \equiv a$(mod $n$)
			\end{flushleft}
		\end{minipage}
		и
	 	\begin{minipage}{0.2\textwidth}
	 		\begin{flushright}
	 			$x \equiv b$(mod $m$)	
	 		\end{flushright}
 		\end{minipage}	
	\end{center}
	допускают одновременное решение тогда и только тогда, когда НОД$(n,m)\mid a - b$; если решение
	существует, докажите, что оно уникально по модулю НОК($n,m$).
	\item Используя Задачу 11, докажите, что система
	\begin{center}
		\begin{minipage}{0.2\textwidth}
			\begin{flushleft}
				$x \equiv 5$(mod $6$)
			\end{flushleft}
		\end{minipage}
		и
		\begin{minipage}{0.2\textwidth}
			\begin{flushright}
				$x \equiv 7$(mod $15$)	
			\end{flushright}
		\end{minipage}	
	\end{center}
	не имеет решений.
	\item Если $x \equiv a$(mod $n$), докажите что $x \equiv a$(mod$2n$) или $x \equiv a + n$(mod $2n$).
	\item Некоторое целое число между 1 и 1200 оставляет остатки 1, 2, 6 при делении на 9, 11, 13 соответственно. Что это за число?
	\item 
		\begin{enumerate} 
		\item Найдите целое число, имеющее остатки 1, 2, 5, 5 при делении на 2, 3, 6, 12 соответственно. (Ихин, умер в 717 году. )
		\item Найдите целое число, имеющее остатки 2, 3, 4, 5 при делении на 3, 4, 5, 6 соответственно. (Бхаскара, родился в 1114 году.)
		\item Найдите целое число, имеющее остатки 3, 11, 15 при делении на 10, 13, 17 соответственно. (Региомонтан, 1436-1473.)	
		\end{enumerate} 
	\end{enumerate}	
\end{document}